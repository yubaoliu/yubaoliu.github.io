% Copyright 2013 Christophe-Marie Duquesne <chmd@chmd.fr>
% Copyright 2014 Mark Szepieniec <http://github.com/mszep>
% 
% ConText style for making a resume with pandoc. Inspired by moderncv.
% 
% This CSS document is delivered to you under the CC BY-SA 3.0 License.
% https://creativecommons.org/licenses/by-sa/3.0/deed.en_US

\startmode[*mkii]
  \enableregime[utf-8]  
  \setupcolors[state=start]
\stopmode

\setupcolor[hex]
\definecolor[titlegrey][h=757575]
\definecolor[sectioncolor][h=397249]
\definecolor[rulecolor][h=9cb770]

% Enable hyperlinks
\setupinteraction[state=start, color=sectioncolor]

\setuppapersize [A4][A4]
\setuplayout    [width=middle, height=middle,
                 backspace=20mm, cutspace=0mm,
                 topspace=10mm, bottomspace=20mm,
                 header=0mm, footer=0mm]

%\setuppagenumbering[location={footer,center}]

\setupbodyfont[11pt, helvetica]

\setupwhitespace[medium]

\setupblackrules[width=31mm, color=rulecolor]

\setuphead[chapter]      [style=\tfd]
\setuphead[section]      [style=\tfd\bf, color=titlegrey, align=middle]
\setuphead[subsection]   [style=\tfb\bf, color=sectioncolor, align=right,
                          before={\leavevmode\blackrule\hspace}]
\setuphead[subsubsection][style=\bf]

\setuphead[chapter, section, subsection, subsubsection][number=no]

%\setupdescriptions[width=10mm]

\definedescription
  [description]
  [headstyle=bold, style=normal,
   location=hanging, width=18mm, distance=14mm, margin=0cm]

\setupitemize[autointro, packed]    % prevent orphan list intro
\setupitemize[indentnext=no]

\setupfloat[figure][default={here,nonumber}]
\setupfloat[table][default={here,nonumber}]

\setuptables[textwidth=max, HL=none]

\setupthinrules[width=15em] % width of horizontal rules

\setupdelimitedtext
  [blockquote]
  [before={\setupalign[middle]},
   indentnext=no,
  ]


\starttext

\subsection[yubao-liu-刘玉宝]{\useURL[url1][http://blog.yubaoliu.cn/resume/][][Yubao
Liu]\from[url1] 刘玉宝}

\placetable[none]{}
\starttable[|l|l|l|l|l|l|]
\HL
\NC Tel:
\NC (+1)570 362-6085
\NC We-chat:
\NC yubao_liu
\NC Japnese:
\NC JLPT-N1 (120)
\NC\AR
\NC E-mail:
\NC yubaoliu@outlook.com
\NC Birthday:
\NC 19891028
\NC English:
\NC CET-6 (456)
\NC\AR
\HL
\stoptable

\subsection[education-training]{Education & Training}

\thinrule

201810-202110, {\bf Doctor Degree Candidate}, {\bf Computer Science and
Engineering}, {\bf Toyohashi University of Technology},
\useURL[url2][http://www.aisl.cs.tut.ac.jp/][][Active Intelligent
Systems Laboratory]\from[url2], Aichi, Japan

201710-201808, {\bf Professional Japanese Language Training},
{\bf Northeast Normal University}, {\bf the Training Center of Ministry
of Education For Studying Overseas}, Jilin, China

201209-201506, {\bf Master of engineering, College of Information
Engineering, Capital Normal University}, Highly Reliable Embedded
System, GPA3.4/4.0, Beijing, China

200809-201206, {\bf Bachelor of Science, College of Computer Science,
Qufu Normal University}, Computer Science and Technology, GPA 3.4/4.0,
Shandong, China

\subsection[work-experience]{Work Experience}

\thinrule

201608-201809, {\bf Senior Software Engineer, iSoftStone Information
Technology(Group)Co.,Ltd.}, Beijing, China

Cooperate with Lenovo Research Center, and proceed project development
and academic research.

{\bf Project}: {\bf Personal Computing Augmented Reality(AR) Glass
Development}, 1.5 years

Develop lightweight AR glass (small size, light weight, suitable for
personal PCs), which is for office use. It is more competitive than
others. The prototype of it has completed. What I am responsibled mainly
includes: * {\bf (Inertial Measurement Unit) IMU}: use IMU to estimate
the pose of AR glass, evaluate and make decision for IMU algorithms:IMU
and magnetic sensor calibration, data fusion. Design tracking demo use
IMU in Unity3D.

\startitemize
\item
  {\bf Tracking}: in order to locate the original poit of 3D world,
  proceed target recognition or marker tracking using OpenCV, EasyAR,
  Kudan, Wikitude, and Vuforia
\item
  {\bf (Simultaneous Localization and Mapping)SLAM}: SLAM solution
  selection and application development using ORBSlam, DSO, Vuforia
  extended tracking, EasyAR, and Kudan SLAM
\item
  {\bf Native SDK}: firmware on STM32 MCU is programed with C and the
  SDK(talk with firmware)on Windows is programed with CPP, and the SDK
  offers API called by Unity3D C\#.
\item
  {\bf GUI HCI design}: C\# control panel for monitoring and controlling
  AR glass hardware, adjust parameters of IMU, Camera, Sensors and
  upgrade firmware version
\item
  {\bf Unity3D AR} main program is programmed in Unity3D with C\#,
  including glass tracking demo using IMU, Vuforia extended tracking and
  some other SLAMs
\item
  {\bf Window System Service}: AR glass plug and play using C\#
\stopitemize

{\bf Project}: {\bf Python Spider}, 2 Months Proceed python spider in
the aid of Scrapy, Xpath, Regular Expression(RE) to grap data like user
comments, articles. And then save these data to database (eg.MongoDB) to
conduct further data analysis.

{\bf Project}: {\bf Voice Meeting Software}, 2 Months To develop a
Windows APP for voice meeting. Evaluate and verify the possible
technologies and write demo codes(C\# WPF) for: - hardware device plug
and play on Windows - record mic device and do voice translate - record
sound card device and do voice translate Voice engigen is implemented in
the aid of Baidu/KeDaXunFei voice engine, etc..

{\bf Project}: {\bf Little Toy Robot} To develop a comany robot, equiped
with simple face expression identification and voice recognition using
Raspberry Pi, Linux, Python.

\startitemize[packed]
\item
  Keyword: AR HMD glass, Computer Vision, SLAM, Unity3D/c/c++/C\#,
  Android/Linux
\stopitemize

\thinrule

201506-201606, {\bf Software Engineer, Intel China Research Center Ltd.
IT FLEX}, Beijing, China

Imaging, Computer Vision; USB (USB3 and XHCI) Pre-silicon verification;
Perl/Python/shell scripts with regular expression

\startitemize
\item
  {\bf USB Pre-silicon Verification}: cooperate with Folsom, California
  USA Team to debug and validate USB and XHCI new features in the next
  generation CPU (e.g.~ice lake) via reviewing USB3 and XHCI
  specification again and again and writing functional coverage and test
  cases using system Verilog for validation on USB simulation platform
  based on Linux
\item
  {\bf Automotive scripts and apps}: text processing with regular
  expression and writing scheduled automation programs; grepping useful
  message from larger files quickly and automatically; wring automotive
  scripts using Perl/Python/Shell to generate Verilog or c/cpp source
  code using predefined format
\item
  {\bf Computer Vision for Intelligent Robot}: cooperate with Intel Lab
  to do code optimization for smart robot which can follow people
  through detecting human via Real Sense Camera
\item
  Keyword: Real Sense Camera, Visual Studio,
  C/C++/Perl/Shell/Python/System Verilog, Cmake, OpenCV, Linux, ace,
  Emacs, Regular Expression
\stopitemize

\subsection[internship-experience]{INTERNSHIP EXPERIENCE}

\thinrule

201502-201506, {\bf Software Engineer, Intel China Research Center Ltd.
IT FLEX}, Beijing, China

Face Beauty App on Android Telephone; image processing; Rewriting OpenCV
basement API function to native API in Android(NDK/JNI) app so as to
make it lightweight

\thinrule

201302-201403, {\bf Next Generation Network Laboratory, Tsinghua
University}, Beijing, China

Linux TCP/IP IPv4 and IPv6 server development; \quotation{Control
Electric Socket Remotely by Telephone in WiFi IPv4&6 Network}; Embedded
ARM Linux programming and ZigBee (uIP (RPL, CoAP, 6LoWPAN), and Contiki)

\thinrule

201207-201209, {\bf R&D Department, Beijing Cyb-Bot Technology
(Cyb-Bot)}, Beijing, China

IOT teaching and research device (Super-IOT), Ember Znet ZigBee and TI
ZigBee protocol, WiFi, STM MCU driver programming, WSN and ARM Linux

\thinrule

201203-201206, {\bf R&D Department, Beijing Universal Pioneering
Technology (UP-TECH)}, Beijing, China

Program robot teaching device to control Industrial Robots using ARM

\subsection[academic]{Academic}

{[}1{]} RTS-vSLAM: Real-time Visual Semantic Tracking and Mapping under
Dynamic Environments, 2020, under review,
\useURL[url3][https://www.youtube.com/watch?v=IP_A_mhHP7Q][][demo]\from[url3]

{[}2{]} Chen, W., Liu, Y. & Wang, H.
{\bf \useURL[url4][http://link.springer.com/article/10.1007\%2Fs12083-014-0303-1][][On
Storage Partitioning of Internet Routing Tables: A P2P-based Enhancement
for Scalable Routers]\from[url4]}{[}J{]}. SCI, Peer-to-Peer Networking
and Applications, 2015, 8(6):952-964

{[}3{]} Liu Y., Chen W.~
{\bf \useURL[url5][https://link.springer.com/chapter/10.1007/978-3-662-46826-5_9][][Multicast
Storage and Forwarding Method for Distributed Router]\from[url5]}
{[}C{]}. EI, Internet Conference of China. Springer, Berlin, Heidelberg,
ICoC 2014:Frontiers in Internet Technologies pp106-117

{[}3{]} 刘玉宝, 陈文龙.
\useURL[url6][http://www.shcas.net/jsjyup/pdf/2015/8/\%E5\%88\%86\%E5\%B8\%83\%E5\%BC\%8F\%E7\%BB\%84\%E6\%92\%AD\%E8\%B7\%AF\%E7\%94\%B1\%E5\%99\%A8\%E5\%AD\%98\%E5\%82\%A8\%E4\%B8\%8E\%E8\%BD\%AC\%E5\%8F\%91\%E4\%BC\%98\%E5\%8C\%96\%E6\%A8\%A1\%E5\%9E\%8B.pdf][][分布式组播路由器存储与转发优化模型]\from[url6]{[}J{]},
计算机应用与软件,2015, 32(8)

\subsection[others]{Others}

\startitemize[packed]
\item
  Facebook:
  \useURL[url7][https://facebook.com/yubaoliu89]\from[url7]\crlf
\item
  BiliBili:
  \useURL[url8][https://space.bilibili.com/52620240]\from[url8]
\item
  Youtube:
  \useURL[url9][https://www.youtube.com/channel/UCqZrQadBvV7-gbt5cc6yCtQ]\from[url9]
\item
  Blog:
  \useURL[url10][http://blogger.yubaoliu.cn][][blogger.yubaoliu.cn]\from[url10]
\stopitemize

\stoptext
